\documentclass[b5paper]{report}
\usepackage{mytemplate/mytemplate}

\usepackage{titlesec}
\titleformat*{\section}{\LARGE\bfseries}
\titleformat*{\subsection}{\large\bfseries}

\usepackage{float}
\usepackage{caption}
\usepackage{indentfirst}


\begin{document}

\title{Notes of Choice under Uncertainty}
\author{Hsun}
\date{January 8, 2026}
\maketitle


\newpage

\begin{abstract}
    This notes follows MWG in preparation of micro-theory II course. Mainly focus on the part of uncertainty. Following the syllabus of Professor Park, we focus on the chapter 6, 10 (Partial equilibrium), 16, 19, 13 with this specific ordering.
\end{abstract}

\newpage

\section*{Ch6: Choice under uncertainty}

\hfill

\subsection*{6.B Expected utility theory}

First, we introduce the notation and remarks.

\begin{notebox}{Notations}
    \begin{itemize}
        \item $C$: the set of all possible outcomes.
        \item $X$: the set of consumption bundles.
    \end{itemize}
\end{notebox}

\noindent \textbf{Remarks:} 
\begin{itemize}
    \item In this section, we consider $C$ to be finite.
    \item We assume that the probabilities of the outcomes from any chosen alternative are \textit{objectively} known.
\end{itemize}

\begin{definition}{6.B.1}
    (\textit{Simple lottery}) A \textit{simple lottery} $L$ is a list $L = (p_1, ..., p_N)$ with $p_n \geq 0$ for all n and $\sum_n p_n = 1$.
\end{definition}

\begin{definition}{6.B.2}
    (\textit{Compound lottery}) Given K simple lotteries $L_k = (p_1^k, ..., p_N^k)$ and probabilities $\alpha_k \geq 0$ with $\sum_k \alpha_k = 1$, the \textit{compound lottery} $(L_1, ..., L_K; \alpha_1, ..., \alpha_K)$ is the risky alternative that yields the simple lottery $L_k$ with probability $\alpha_k$.
\end{definition}

In a simple lottery, the outcomes that may result are certain. For any compound lottery $(L_1, ..., L_K; \alpha_1, ..., \alpha_K)$, we can calculate a corresponding \textit{reduced lottery} as the simple lottery $\mathrm{L} = (p_1, ..., p_N)$ that generates the same ultimate distribution over outcomes, where $p_n = \alpha_1 p_n^1 + ... + \alpha_Kp_n^K$. Therefore, $L = \alpha_1 L_1 + ... + \alpha_K L_K$.

\subsubsection*{Preference over lotteries}

We assume that \textit{for any risky alternative, only the reduced lottery over final outcomes is relevant to the decision maker.}\footnote{This is the reduction axiom in the notes of investment theory.} Therefore, we let $\mathcal{L}$ to be the set of all simple lotteries over the set of outcomes $C$. 

\begin{definition}{6.B.3}
    (\textit{Continuity axiom}) The preference relation $\succcurlyeq$ on the space of simple lotteries $\mathcal{L}$ is \textit{continuous} if or any $L, L', L'' \in \mathcal{L}$, the sets
    \begin{align*}
        &\{\alpha \in [0,1]: \alpha L + (1-\alpha)L' \succcurlyeq L'' \} \subset [0,1],\\
        &\{\alpha \in [0,1]: L'' \succcurlyeq \alpha L + (1-\alpha)L' \} \subset [0,1]
    \end{align*}
    are closed.
\end{definition}

The continuous axiom therefore implies that small changes in probabilities do not change the ordering between two lotteries. Moreover, it excludes the case that the decision maker has lexicographic preferences for alternatives with a zero probability of some outcome. As in ch3, the continuity axiom implies the existence of a utility function representation.\footnote{See prop. 3.C.1: Suppose the rational $\succcurlyeq$ on $X$ is continuous, then there is a continuous utility function $u(x)$ that represents $\succcurlyeq$.}

\begin{definition}{6.B.4}
    (\textit{Independence axiom}) The preference relation $\succcurlyeq$ on $\mathcal{L}$ satisfies the \textit{independence axiom} if for all $L, L', L'' \in \mathcal{L}$ and $\alpha \in (0,1)$ we have
    \begin{align*}
        L \succcurlyeq L'\quad \text{if and only if}\quad \alpha L + (1-\alpha)L'' \succcurlyeq \alpha L' + (1-\alpha)L''
    \end{align*}
\end{definition}

The intuition behind the independence axiom is that the preference ordering preserves even if we mix the two lotteries with a third one. Note this axiom is the most important but controversial part of the EU theory.

\begin{definition}{6.B.5}
    (\textit{VNM utility function}) The utility function $U: \mathcal{L} \rightarrow \mathbb{R}$ has a \textit{VNM utility function} form if there is an assignment of numbers $(u_1, ..., u_N)$ to the $N$ outcomes in $C$ such that for any simple lottery $L = (p_1, ..., p_N) \in \mathcal{L}$ we have
    \begin{align*}
        U(L) = u_1p_1 + ... + u_Np_N
    \end{align*}
\end{definition}

\begin{proposition}{6.B.1}
    A utility function $U: \mathcal{L} \rightarrow \mathbb{R}$ has an expected utility form if and only if it is \textit{linear}. That is, if and only if it satisfies 
    \begin{align*}
        U\left(\sum_{k=1}^K \alpha_k L_k\right) = \sum_{k=1}^K \alpha_k U(L_k)
    \end{align*}
    for any $K$ lotteries $L_k\in \mathcal{L}$ and probabilities ($\alpha_1, ..., \alpha_K) \geq 0$ with $\sum_{k=1}^K \alpha_k = 1$.
\end{proposition}

\begin{pf}
    \textbf{Proof:} $(\Rightarrow)$ Suppose $U$ has the expected utility form, then for any compound lottery $(L_1, ..., L_K; \alpha_1, ..., \alpha_K)$, where $L_k = (p_1^k, ..., p_N^k)$, its reduced lottery is $L' = \sum_k \alpha_kL_k$. Hence,
    \begin{align*}
        U\left(\sum_{k=1}^K \alpha_k L_k\right) = \sum_{n=1}^N u_n\left(\sum_{k=1}^K \alpha_k p_n^k\right) = \sum_{k=1}^K \alpha_k\left(\sum_{n=1}^N u_n p_n^k\right) = \sum_{k=1}^K \alpha_k U(L_k) 
    \end{align*}
    \hfill $\square$\\
    $(\Leftarrow)$ Suppose $U$ satisfies the equality, we can write any $L = (p_1, ...,p_N)$ as a convex combination of the degenerate lotteries $(L^1, ..., L^N)$. We then have $U(L) = U(\sum_n p_n L^n) = \sum_n p_n U(L^n) = \sum_n p_n u_n$, which is the expected utility form.

    \hfill $\square$
\end{pf}

By the above proposition, we know that the expected utility property is a \textit{cardinal} property of utility functions defined on the space of lotteries.

\subsubsection*{Expected utility theorem}
 From above discussion, we know that if a decision maker's preference over lotteries satisfies the axioms, then its preference can be represented by an VNM utility function. (the proof is omitted) We further discuss the properties of indifference curves.\\

 Since the expected utility form is linear in probabilities, the representability by the expected utility form is equivalent to the indifference curves being straight and parallel lines. In particular, this is the result of independence axiom. For example,  

\begin{figure}[H]
\centering
\includegraphics[width=0.9\textwidth]{figs/fig.6.b.5.png}
\end{figure}

\subsection*{6.C Money lotteires and Risk aversion}

Let continuous varaible $x$ denote the amount of money and the CDF $F: \mathbb{R}\ \rightarrow\ [0,1]$ be the monetary lottery. Also, we take the lotttery space $\mathcal{L}$ to be the set of all CDF over non-negative amounts of money.

\subsubsection*{Risk aversion}

\begin{definition}{6.C.1}
    (\textit{Risk preference}) A decision maker is \textit{risk averse} if for any lottery $F(\cdot)$,
    the degenerate lottery that yields the amount $\int x dF(x)$ with certainty is at least as good
    as the lottery $F(\cdot)$ itself.\\

    If the decision maker is always indifferent between these two lotteries, we say that he is \textit{risk neutral}.\\
    
    Finally, we say that one is \textit{strictly risk averse} if indifference holds only when the two lotteries are the same (i.e., when $F(\cdot)$ is degenerate).
\end{definition}

If one's preference admits an expected utility form, then one is risk averse if and only if 
\begin{align*}
    \int u(x) dF(x) \leq u \left( \int x dF(x) \right) \quad \text{for all}\ F(\cdot).
\end{align*}
The inequality is called the \textit{Jenson's inequality}. Therefore, we see that (strict) risk aversion is equivalent to the (strict) concavity of $u(\cdot)$.

\begin{definition}{6.C.2}
    (\textit{Certainty equivalent}) Given $u(\cdot)$, we define 
    \begin{itemize}
        \item The \textit{certainty equivalent} of $F(\cdot)$, $c(F,u)$, is the amount of money for which the agent is indifferent between the gable $F(\cdot)$ and the certain amount $c(F,u)$. That is,
        \begin{align*}
            u(c(F,u)) = \int u(x) dF(x)
        \end{align*}
        \item For any fixed amount of money $x$ and $\epsilon>0$, the \textit{probability premium}, $\pi(x,\epsilon,u)$, is the excess in winning probability ovver fair odds that makes the individual indifferent between the certain outcome $x$ and a gamble betweeen the two outcomes $x + \epsilon$ and $x - \epsilon$. That is, 
        \begin{align*}
            u(x) = (\frac{1}{2} + \pi(x,\epsilon,u))u(x + \epsilon) + (\frac{1}{2} - \pi(x,\epsilon,u))u(x - \epsilon)
        \end{align*}
    \end{itemize}
\end{definition}

\begin{proposition}{6.C.1}
    Suppose a decision maker is an expected utility maximizer with $u(\cdot)$ on amounts of money, then the following are quivalent:
    \begin{itemize}
        \item[(i)] The decision maker is risk averse
        \item[(ii)] $u(\cdot)$ is concave
        \item[(iii)] $c(F,u) \leq \int x dF(x)$ for all $F(\cdot)$
        \item[(iv)] $\pi(x,\epsilon,u) \geq 0$ for all $x$ and $\epsilon \geq 0$   
    \end{itemize}
\end{proposition}

\begin{pf}
    \textbf{Proof:} We only show that (ii) $\Leftrightarrow$ (iii), (i) $\Rightarrow$ (iv), and (iv) $\Rightarrow$ (ii).\\

    (ii) $\Leftrightarrow$ (iii): For any $F(\cdot)$,
    \begin{align*}
        &c(F,u) \leq \int xdF(x) \\
        \Leftrightarrow\ &u(c(F,u)) \leq u\left(\int xdF(x) \right)\\
        \Leftrightarrow\ &\int u(x)dF(x) \leq u\left(\int xdF(x) \right) \tag*{(def. 6.C.2)}
    \end{align*}
    \hfill $\square$\\
    (i) $\Rightarrow$ (iv): Consider a risk averse decision maker. For any $x$ and $\epsilon \geq 0$, there exists a $\pi$ s.t. 
    \begin{align*}
        u(x) 
        &= (\frac{1}{2} + \pi)u(x + \epsilon) + (\frac{1}{2} - \pi)u(x - \epsilon) \\
        &\geq \frac{1}{2}u(x + \epsilon) + \frac{1}{2}u(x - \epsilon)
    \end{align*}
    Rearrange the inequality, we obtain
    \begin{align*}
        \pi u(x+\epsilon) \geq \pi u(x-\epsilon)
    \end{align*}
    Since $u'(\cdot) \geq 0$, we know $u(x+\epsilon) \geq u(x-\epsilon)$ and hence $\pi \geq 0$ for any $x$ and $\epsilon \geq 0$.
    
    \hfill $\square$ \\
    (iv) $\Rightarrow$ (ii): We claim that (iv) implies $u(\frac{1}{2}x + \frac{1}{2}y) \geq \frac{1}{2}u(x) + \frac{1}{2}u(y)$ for any $x,y$.\\

    Suppose $\pi \geq 0$ for all $x$ and $\epsilon \geq 0$ and $\exists\ x,y$ s.t. $u(\frac{1}{2}x + \frac{1}{2}y) < \frac{1}{2}u(x) + \frac{1}{2}u(y)$. WLOG we assume $x \geq y$.
\end{pf}

    \newpage

\begin{pf}
    Take $\epsilon = \frac{1}{2}x - \frac{1}{2}y \geq 0$. There exists a $\pi \geq 0$ s.t.
    \begin{align*}
        u(\frac{1}{2}x + \frac{1}{2}y) 
        &= (\frac{1}{2} + \pi)u(\frac{1}{2}x + \frac{1}{2}y + \epsilon) + (\frac{1}{2} - \pi)u(\frac{1}{2}x + \frac{1}{2}y - \epsilon)\\
        &= (\frac{1}{2} + \pi)u(x) + (\frac{1}{2} - \pi)u(y)
    \end{align*}
    Since $x \geq y$, we conclude that $u(\frac{1}{2}x + \frac{1}{2}y) \geq \frac{1}{2}u(x) + \frac{1}{2}u(y)$ for any $x,y$. Thus, $u(\cdot)$ is concave.
    
    \hfill $\square$
\end{pf}

\subsubsection*{The measurement of risk aversion}

\begin{definition}{6.C.3}
    (\textit{Arrow-Pratt measure}) Given a twice-differentiable $u(\cdot)$ for money, the \textit{Arrow-Pratt coefficient of absolute risk aversion} at $x$ is defined as 
    \begin{align*}
        r_A(x) = -\frac{u''(x)}{u'(x)}
    \end{align*}
\end{definition}

One motivation behind the measure is to fix a wealth level $x$ and study the probability premium $\pi(x, \epsilon, u)$ as $\epsilon \rightarrow 0^+$. By def 6.C.2, for any $x$ and $\epsilon \geq 0$, the probability premium is defined as $\pi(\epsilon)$ s.t. 
\begin{align*}
    u(x) = (\frac{1}{2} + \pi(\epsilon))u(x + \epsilon) + (\frac{1}{2} - \pi(\epsilon))u(x - \epsilon)
\end{align*}
Differentiate both sides w.r.t. $\epsilon$ twice, and take $\epsilon \rightarrow 0^+$, we obtain
\begin{align*}
    0 = 4\pi'(0)u'(x) + u''(x)
\end{align*}
Hence, 
\begin{align*}
    r_A(x) = 4\pi'(0)
\end{align*}
This means that $r_A(x)$ measures \textit{the rate at which the probability premium increases at certainty with small risk.}\\

To compare the level of risk aversion across agents, the following criterion seem feasible:
\begin{itemize}
    \item[(i)] $r_A(x, u_2) \geq r_A(x, u_1)$ for all $x$
    \item[(ii)]  There exists an increasing concave function $\psi(\cdot)$ such that $u_2(x) = \psi(u_1(x))$ at all $x$. In other words, $u_2(\cdot)$ is “more concave” than $u_1(\cdot)$
    \item[(iii)] $c(F,u_2) \leq c(F,u_1)$ for any $F(\cdot)$
    \item[(iv)] $\pi(x,\epsilon, u_2) \geq \pi(x, \epsilon, u_1)$ for any $x$ and $\epsilon \geq 0$
    \item[(v)] Whenever $u_2$ finds a lottery $F(\cdot)$ at least as good as a riskless outcome $\bar{x}$, then $u_1$ also finds $F(\cdot)$ at least as good as $\bar{x}$.
\end{itemize}

\begin{proposition}{6.C.2}
    Criterion (i) to (v) are equivalent.
\end{proposition}
\begin{pf}
    \textbf{Proof:} We only show (i) $\Leftrightarrow$ (ii).\\

    We always have some increasing $\psi(\cdot)$ s.t. $u_2(x) = \psi(u_1(x))$ as they are ordinally identical after the transformation. Differentiate both sides w.r.t. $x$, we obtain
    \begin{align*}
        u'_2(x) = \psi'(u_1(x)) \cdot u'_1(x)
    \end{align*}
    and 
    \begin{align*}
        u''_2(x) = \psi''(u_1(x)) \cdot [u'_1(x)]^2 + \psi'(u_1(x))\cdot u''_1(x)
    \end{align*}
    Dividing both sides by $u'_2(x) > 0$, 
    \begin{align*}
        r_A(x,u_2) 
        &= -\frac{u''_2(x)}{u'_2(x)} \\
        &= -\frac{\psi''(u_1(x)) \cdot [u'_1(x)]^2 + \psi'(u_1(x))\cdot u''_1(x)}{\psi'(u_1(x)) \cdot u'_1(x)}\\
        &= r_A(x,u_1) - \frac{\psi''(u_1(x))}{\psi'(u_1(x))}u'_1(x)
    \end{align*}
    Hence, 
    \begin{align*}
        r_A(x,u_2) \geq r_A(x,u_1)\ \Leftrightarrow\ \psi''(u_1(x)) \leq 0\ \text{for all}\ x
    \end{align*}
    \hfill $\square$
\end{pf}

\noindent \textbf{Reamrk:} Note that the "more-risk-averse-than" relation is transitive but \textit{not complete}. That is, we generally have $r_A(x,u_1) > r_A(x,u_2)$ for some $x$ but $r_A(x,u_1) < r_A(x,u_2)$ for some other $x'$.

\begin{definition}{6.C.4}
    $u(\cdot)$ for money exhibits \textit{decreasing absolute risk aversion} if $r_A(x,u)$ is a decreasing function of $x$.
\end{definition}

\begin{proposition}{6.C.3}
    The following are equivalent:
    \begin{itemize}
        \item[(i)] $u(\cdot)$ exhibits decreasing absolute risk aversion
        \item[(ii)] For $x_2 < x_1$, $u_2(z) = u(x_2+z)$ is a concave transformation of $u_1(z) = u(x_1+z)$
        \item[(iii)] For any rsik $F(z)$, the certainty equivalent formed by adding risk $z$ at wealth level $x$, denoted $c_x$ is s.t. $(x - c_x)$ is decreasing in $x$. That is, the higher the wealth level $x$ is, the less one is willing to pay to get rid of the risk
        \item[(iv)] The probability premium $\pi(x,\epsilon,u)$ is decreasing in $x$
        \item[(v)] For any $F(z)$, if $\int u(x_2+z)dF(z) \geq u(x_2)$ and $x_2 < x_1$, then $\int u(x_1+z)dF(z) \geq u(x_1)$
    \end{itemize}
\end{proposition}

\begin{pf}
    \textbf{Proof:} This is the result of prop. 6.C.2.

    \hfill $\square$
\end{pf}

From prop. 6.C.3, we know the underlying wealth level has significant effects on the risk averse behavior toward risk. Therefore we introduce another measure of risk aversion capturing the relative risk at a certain wealth level.

\begin{definition}{6.C.5}
    (\textit{Relative risk aversion}) Given $u(\cdot)$, the \textit{coefficient of relative risk aversion} at $x$ is 
    \begin{align*}
        r_R(x,u) = -\frac{xu''(x)}{u'(x)}
    \end{align*}
\end{definition}

Note the requirement of \textit{non-increasing relative risk aversion} is stronger than that of decreasing absolute risk aversion we studied preveiously. This is because $r_R(x,u) = x r_A(x,u)$. A risk averse agent with non-increasing relative risk aversion exhibits decreasing absolute risk aversion, but the converse is not true. The following proposition mimics prop. 6.C.3.

\begin{proposition}{6.C.4}
    The following conditions on $u(\cdot)$ are equivalent:
    \begin{itemize}
        \item[(i)] $r_R(x,u)$ is decreasing in $x$
        \item[(ii)] For $x_2 < x_1$, $\tilde{u}_2(x) = u(tx_2)$ is a concave transformation of $\tilde{u}_1(x) = u(tx_1)$
        \item[(iii)] Given any risk $F(t)$ on $t>0$, the certainty equivalent $\bar{c}_x$ defined by $u(\bar{c}_x) = \int u(tx)dF(t)$ is s.t. $\frac{x}{\bar{c}_x}$ is decreasing in $x$
    \end{itemize}
\end{proposition}
\begin{pf}
    \textbf{Proof:} We show (i) implies (iii).\\

    Fix $F(t)$ on $t>0$. For any $x$, define $u_x(t) = u(tx)$ and let $c(x)$ be the usual certainty equivalent s.t. $u_x(c(x)) = \int u_x(t)dF(t)$\\

    Note that 
    \begin{align*}
        -\frac{u''_x(t)}{u'_x(t)} 
        &= -x \frac{u''(tx)}{u'(tx)}\\
        &= -(\frac{1}{t})tx\frac{u''(tx)}{u'(tx)}\\
        &= (\frac{1}{t})\cdot r_R(tx,u) \ \text{for any}\ x
    \end{align*}

    If (i) holds, then $u_{x'}(\cdot)$ is less risk averse than $u_{x''}(\cdot)$ for any $x' > x$. Thus, by prop. 6.C.2, $c(x') > c(x)$ and we conclude $c(\cdot)$ is increasing. We know then
    \begin{align*}
        u_x(c(x)) &= u(xc(x))\\
        &= \int u_x(t)dF(t) = \int u(tx)dF(t) = u(\bar{c}_x) 
    \end{align*}
    Hence, $\dfrac{\bar{c}_x}{x} = c(x)$. We conclude that $\dfrac{x}{\bar{c}_x}$ is decreasing as desired. 

    \hfill $\square$
\end{pf}

\subsection*{6.D Comparison of payoff distributions in terms of return and risk}

In previous section, we discuss on how to compare utility functions. While in this section, we would like to compare payoff distributions. Specifically, we would like to contruct formal ways to do such a comparison w.r.t. our two major concerns: return and the dipersion of return. For the following specifications, we restrict ourselves to distributions $F(\cdot)$ s.t. $F(0) = 0$ and $F(x) = 1$ for some $x$.\\

To see whether the distribution $F(\cdot)$ yields unambiguously higher returns than $G(\cdot)$, we may check two things:
\begin{itemize}
    \item[(i)] Whether every EU-maximizer who values more than less prefers $F(\cdot)$ to $G(\cdot)$
    \item[(ii)] For every amount of money $x$, the probability of receiving at least $x$ is higher under $F(\cdot)$ than under $G(\cdot)$
\end{itemize}
Hopefully, it turns out that these two lead to the same concept of the \textit{first-order stochastic dominance}. (They are indeed equivalent!)

\subsubsection*{First-order stochastic dominance}

\begin{definition}{6.D.1}
    (\textit{First-order stochastic dominance}) The distribution $F(\cdot)$ first-order stochastically dominates $G(\cdot)$ if, for every non-decreasing function $u: \mathbb{R} \rightarrow \mathbb{R}$, we have
    \begin{align*}
        \int u(x)dF(x) \geq \int u(x)dG(x)
    \end{align*}
\end{definition}

\begin{proposition}{6.D.1}
    The distribution of monetary payoffs $F(\cdot)$ first-order stochastically dominates $G(\cdot)$ \textit{if and only if} $F(x) \leq G(x)$ for every $x$.
\end{proposition}
\begin{pf}
    \textbf{Proof:} ($\Rightarrow$) Given $F(\cdot)$ and $G(\cdot)$, suppose $F(\cdot)$ first-order stochastically dominates $G(\cdot)$. Take some arbitrary $\bar{x}$, we define a non-decreasing function
    \begin{align*}
        u(x) = \begin{cases}
            1 \quad \text{if}\ x \geq \bar{x}\\
            0 \quad \text{otherwise}
        \end{cases}
    \end{align*}
    By def. 6.D.1, 
    \begin{align*}
        \int u(x)dF(x) \geq \int u(x)dG(x)\ \Rightarrow\ &1 - F(\bar{x}) \geq 1 - G(\bar{x})\\
        \Rightarrow\ &G(\bar{x}) \geq F(\bar{x})
    \end{align*}
    \hfill $\square$ \\
    ($\Leftarrow$) Suppose $F(x) \leq G(x)$ for every $x$. For any non-decreasing differentiable\footnote{We omit the proof that it suffices to show the case of differentiable $u(\cdot)$.} function $u(\cdot)$, we use integration by parts:
    \begin{align*}
        \int u(x) dF(x) - \int u(x) dG(x) &= \int u(x) d[F(x) - G(x)] \\
        &= \underbrace{\left. u(x)[F(x)-G(x)] \right|_{-\infty}^{\infty}}_{=0} - \int [F(x)-G(x)]u'(x)dx \\
        &= \int [G(x)-F(x)]u'(x)dx
    \end{align*}
\end{pf}

\newpage

\begin{pf}
    Since $G(x) \geq F(x)$ and $u'(x) \geq 0$, the integrand is non-negative. Hence, the integral is $\geq 0$.

    \hfill $\square$
\end{pf}

\begin{figure}[H]
\centering
\includegraphics[width=0.4\textwidth]{figs/fig.6.d.2.png}
\caption*{$F(\cdot)$ first-order stochastically dominates $G(\cdot)$}
\end{figure}

One thing worth noting is that although $F(\cdot)$ first-order stochastically dominating $G(\cdot)$ implies the mean return of $F(\cdot)$ is greater than that of $G(\cdot)$, the converse generally isn't true. For a counterexample, consider
\begin{align*}
    F(z) &= \begin{cases}
        0 \quad \text{if}\ z<0\\
        \frac{1}{4} \quad \text{if}\ 0\leq z <2\\
        1 \quad \text{if}\ 2\leq z
    \end{cases} \\
    G(z) &= \begin{cases}
        0 \quad \text{if}\ z<1\\
        1 \quad \text{otherwise}
    \end{cases}
\end{align*}
We may check that $F(\frac{1}{2}) = \frac{1}{4} > 0 = G(\frac{1}{2})$. Thus, $F(\cdot)$ does not first-order stochastically dominate $G(\cdot)$ by prop.6.D.1. However,
\begin{align*}
    \mathrm{E}[F(z)] = 0 \cdot \frac{1}{4} + 2 \cdot \frac{3}{4} = \frac{3}{2} > \mathrm{E}[G(z)] = 1
\end{align*}
\hfill $\square$

\subsubsection*{Second-order stochastic dominance}

For the discussion of second-order stochastic dominance, we restrict ourselves to distributions with the same mean.

\begin{definition}{6.D.2}
    (\textit{Second-order stochastic dominance}) For any two distributions $F(\cdot)$ and $G(\cdot)$ with the same mean, $F(\cdot)$ \textit{second-order stochastically dominates} (or is \textit{less risky than}) $G(\cdot)$ if, for every non-decreasing \textit{concave} function $u: \mathbb{R} \rightarrow \mathbb{R}$, we have
    \begin{align*}
        \int u(x)dF(x) \geq \int u(x)dG(x)
    \end{align*}
\end{definition}

Now we introduce another way to characterize the second-order stochastic dominance, known as \textit{mean-preserving spreads}.\\

Consider a compound lottery that yields $x$ according to the distribution $F(\cdot)$ in the first stage, and randomize all possible outcomes of $x$ so that the final payoff is $x+z$ in the second stage, where $z$ follows a distribution $H_x(z)$ with zero mean.

\begin{definition}{6.D.1*}
    (\textit{Mean-preserving spreads}) Let $G(\cdot)$ denote the reduced lottery resulting from the process we described above. If a distribution can be obtained from lottery $F(\cdot)$ in such a process for some distribution $H_x(\cdot)$ (e.g. $G(\cdot)$), then we say it is a \textit{mean-preserving spread} of $F(\cdot)$.
\end{definition}

To associate the mean-preserving spread with def. 6.D.1, we need to check two things: 

\begin{itemize}
    \item[(i)] $F(\cdot)$ and $G(\cdot)$ share the same mean
    \item[(ii)] For any non-decreasing concave function $u: \mathbb{R} \rightarrow \mathbb{R}$, the inequality in def. 6.D.1 holds.
\end{itemize}

\begin{pf}
    \textbf{Check:}\\
    (i):
    \begin{align*}
        \mathrm{E}[G(x)] = \int \left(\int (x+z) dH_x(z) \right) dF(x) = \int x dF(x) = \mathrm{E}[F(x)]
    \end{align*}
    \hfill $\square$\\

    (ii): Fix a concave function $u: \mathbb{R} \rightarrow \mathbb{R}$,
    \begin{align*}
        \int u(x) dG(x) &= \int \left(\int u(x+z) dG_x(z) \right) dF(x) \\
        &\leq \int u\left( \int (x+z) dG_x(z) \right) dF(x) = \int u(x) dF(x)
    \end{align*}
    \hfill $\square$
\end{pf}

After checking the conditions, we know that def 6.D.1* implies that $F(\cdot)$ second-order stochastically dominates $G(\cdot)$. In later proposition, we will show that the two concepts are essentially equivalent.

\begin{definition}{6.D.1**}
    (\textit{Elementary increase in risk}) $G(\cdot)$ forms an \textit{elementary increase in risk} from $F(\cdot)$ if $G(\cdot)$ is generated from $F(\cdot)$ by transfering the mass $F(\cdot)$ assigns to $[x',x'']$ to the endpoint $x'$ and $x''$ s.t. the mean preserves.
\end{definition}


\begin{proposition}{6.D.2}
    Given two distributions $F(\cdot)$ and $G(\cdot)$ with the same mean. The following are equivalent:
    \begin{itemize}
        \item[(i)] $F(\cdot)$ \textit{second-order stochastically dominates} $G(\cdot)$
        \item[(ii)] $G(\cdot)$ is a \textit{mean-preserving spread} of $F(\cdot)$
        \item[(iii)] $\int_0^x G(t) dt \geq \int_0^x F(t) dt$ for all $x$
        
    \end{itemize}
    
\end{proposition}
\begin{pf}
    \textbf{Proof:} Omitted.
\end{pf}

\subsection*{6.E State-dependent utility (Optional)}

In previous sections, we implicitly characterize the preference of decision makers in a way that the underlying state that leads to the monetary payoff is irrelevant to the eventual payoff. However, there are plenty of cases where people care about the resulting cause that leads to the payoff. For example, one must be injured or sick to receive the insurance payment.\\ 

For later discussion, we refer to the underlying causes as \textit{states} or \textit{states of nature} and denote the finite set of states by $S = \{1, ..., S \}$. For each state $s \in S$, we assume there is a well-defined, objective probability $\pi_s > 0$ assigned and denote the payoff by $x_s \geq 0$. Formally, the uncertain alternative with monetary payoff can be described with a random variable $g: S \rightarrow \mathbb{R}_+$ so that every $g$ can be associated with a distribution function $F(\cdot)$ with $F(x) = \sum_{\{s: g(s) \leq x\}} \pi_s$ for all $x$.

\subsubsection*{State-dependent preferences and the extended EU theory}

\begin{definition}{6.E.2}
    (\textit{Extended expected utility representation}) The preference relation $\succcurlyeq$ has an \textit{Extended expected utility representation} if for every $s \in S$, there is a function $u_s: \mathbb{R}_+ \rightarrow \mathbb{R}$ s.t. for any $(x_1, ..., x_S) \in \mathbb{R}_+^S$ and $(x'_1, ..., x'_S) \in \mathbb{R}_+^S$,
\end{definition}

\begin{definition}{6.E.2}
    (conti.) 
    \begin{align*}
        (x_1, ..., x_S) \succcurlyeq (x'_1, ..., x'_S)\ \text{if and only if}\ \sum_s \pi_s u_s(x_s) \geq \sum_s \pi_s u_s(x'_s)
    \end{align*}
\end{definition}

Under this definition, we allow individuals to have different utility functions in each state, which generalizes the \textit{state-independent} utility function we studied in previous sections.


\begin{figure}[H]
\centering
\includegraphics[width=0.5\textwidth]{figs/fig.6.e.1.png}
\end{figure}

The figure above illustrates a major difference between the state-dependent and the \textit{state-uniform} utilities. We call the set of random varaibles that pay the same amount at every state the \textit{money certainty line}. In a two-state case as the figure shows, the MRS of a state-dependent utlity around the money certainty line is $\frac{\pi_1 u'_1(\bar{x})}{\pi_2 u'_2(\bar{x})}$, which reflects the probability of the occurence of the state and the nature of state dependence. While that of a state-uniform utility is only $\frac{\pi_1}{\pi_2}$. (the rest contents are skipped...)

\subsection*{6.F Subjective probability theory (Optional)}

In many real world cases, the state of the world is not associated with a recognizable, objective probability but rather often is linked to subjective estimates. The \textit{subjective probability theory}, however, states that there are still restrictions imposed on preferences such that decision makers behave \textit{as if} utilities are assigned to outcomes, probabilities are attached to states of nature, and the decisions are made by taking expected utilities.\\

Following the notations in 6.E, the set of states is denoted by $S=\{1, ...,S \}$ while the probabilities are not given.

\newpage


%%%%%%%%%%%%%%%%%%%%%%%%%%%%%%%%%%%%%%%%%%%%%%%%%%%%%%%%%%%%%%%%%%%%%%%%%%%%%%%%%%%%%%%%%%%%%%%%%%%%%%%%%%%%%%

\section*{Ch10: Competitive markets}
\hfill

In this chapter, we will focus on the partial equilibrium analysis and the welfare theorem.

\subsection*{10.B Pareto optimality and Competitive equilibria}

\begin{notebox}{Notations}
    \begin{itemize}
        \item $\omega_l \geq 0:$ endowment of the good indexed by $l = 1,..., L$
        \item $x_i = (x_{1i}, ..., x_{Li}) \in \mathbb{R}^L$: consumption bundle of consumer $i$, $i = 1,..., I$
        \item $y_j = (y_{1j}, ..., y_{Lj}) \in \mathbb{R}^L$: production vector of firm $j$, $j = 1,..., J$
    \end{itemize}
\end{notebox}

\noindent \textbf{Remark:} It is possible for firms to transform some of the initial endowment into the other goods with some technologies. We use negative entries in the production vector to denote input usage, so the total amount of goods $l$ available is $\omega_l + \sum_j y_{lj}$.

\subsubsection*{Pareto optimality}

\begin{definition}{10.B.1}
    (\textit{Feasible allocation}) An economic allocation $(x_1, ..., x_I, y_1, ..., y_J)$ is a specification of a consumption bundle $x_i \in X_i\ \forall$ consumers and a production vector $y_j \in Y_j\ \forall$ firms. The allocation $(x_1, ..., x_I, y_1, ..., y_J)$ is \textit{feasible} if 
    \begin{align*}
        \sum_{i=1}^I x_{li} \leq \omega_l + \sum_{j=1}^J y_{lj}\ \text{for}\ l=1, ..., L
    \end{align*}
\end{definition}


\begin{definition}{10.B.2}
    (\textit{Pareto efficiency}) A feasible allocation $(x_1, ..., x_I, y_1, ..., y_J)$ is \textit{Pareto optimal} (or \textit{Pareto efficient}) if there is no other feasible allocation $(x'_1, ..., x'_I, y'_1, ..., y'_J)$ s.t. $u_i(x'_i) \geq u_i(x_i)$ for all $i$ and $u_i(x'_i) > u_i(x_i)$ for some $i$.
\end{definition}

Following the concept of Pareto efficiency, we define the \textit{utility possibility set} as $U = \{(u_1, ..., u_I) \in \mathbb{R}^I$: there exists a feasible allocation $(x_1, ..., x_I, y_1, ..., y_J)$ s.t. $u_i \leq u_i(x_i)$ for all $i \}$

\subsubsection*{Competitive equilibria}

In this chapter, we are concerned with the analysis of \textit{competitive market economies}. In such an economy, society's initial endowments and technological possibilities (i.e. firms) are owned by consumers. We denote the share of firm $j$ owned by consumer $i$ by $\theta_{ij}$, so that he/she can claim a fraction of $\theta_{ij}$ of firm $j$'s profits. ALso in a competitive economy, a market exists for each goods, and all consumers and producers are price takers.\\

\noindent \textbf{Remark:} Recall that in textbooks, we usually argue that the reason why an equilibrium must involve no excess demand or supply is because, for example, some consumers may be better off to offer a higher price to receive as much goods as desired. In such a justification, we actually made use of the fact that consumers and producers are not simply take the price as given, which may seem to lead to a contradiction.\\

We have to recognize here that consumers and producers both always have the ability to change their offered price. Meanwhile, the price-taking assumption is actually refer to the case that there is \textit{no incentive} for them to alter the price, if taken as given, eqaute the supply and demand. As long as consumers and producers are able to make their desired offer, at the price that equates supply and demand, consumers will not wish to raise the price and producers will not wish to lower it. This answer the "contradiction" we ran in to at the first glance.

\begin{definition}{10.B.3}
    (\textit{Competitive equilibrium}) The allocation $(x^*_1, ..., x^*_I, y^*_1, ..., y^*_J)$ and price vector $p^* \in \mathbb{R}^L$ consititute a \textit{competitive equilibrium} (or \textit{Walrasian equilibrium}) if the following conditions are satisfied:
    \begin{itemize}
        \item[(i)] \textit{Profit maximization}: For each firm $j$, $y^*_j$ solves
        \begin{align*}
            \max_{y_j \in Y_j} p^* \cdot y_j
        \end{align*}
        \item[(ii)] \textit{Utility maximization}: For each consumer $i$, $x^*_i$ solves
        \begin{align*}
            \max_{x_i \in X_i} u_i(x_i)\ s.t.\ p^* \cdot x_i \leq p^* \cdot \omega_i + \sum_{j=1}^{J} \theta_{ij}(p^* \cdot y^*_j)
        \end{align*}
        \item[(iii)] \textit{Market clearing}: For each good $l$, 
        \begin{align*}
            \sum_{i=1}^{I} x^*_{li} = \omega_l + \sum_{j=1}^{J} y^*_{lj}
        \end{align*}
    \end{itemize}
\end{definition}

\begin{exercise}{10.B.2}
    Show that if allocation $(x^*_1, ..., x^*_I, y^*_1, ..., y^*_J)$ and price vector $p^* \gg 0$ form a competitive equilibrium, then allocation $(x^*_1, ..., x^*_I, y^*_1, ..., y^*_J)$ and price vector $\alpha p^*$ also form a competitive equilibrium for any scalar $\alpha > 0$.
\end{exercise}
\begin{pf}
    \textbf{Proof:} As $y^*_1, ..., y^*_J$ solves the profit maximization problem given $p^*$, they solve the profit maximization problem under $\alpha p^*$. Plugging this result to conumsers' budget constraints, the utility maximization problem becomes:
    \begin{align*}
        \max_{x_i \in X_i} u_i(x_i)\ s.t.\ \alpha p^* \cdot x_i \leq \alpha p^* \omega_i + \sum_{j=1}^{J} \theta_{ij} (\alpha p^* \cdot y^*_j)
    \end{align*}
    Since $\alpha > 0$, the utility maximization problem is identical to the original one. Thus, the same $x_1,..., x_I$ solve the utility maximization problem under $\alpha p^*$.\\

    Finally, as the solution to the two maximization problems,$x^*_i$ and $y^*_j$, are unchanged, the market clearing condition holds naturally. 

    \hfill $\square$
\end{pf}

\begin{lemma}{10.B.1}
    (\textit{Walras' law}) If the allocation $(x_1, ..., x_I, y_1, ..., y_J)$ and the price vector $p \gg 0$ satisfy the market clearing condition for all goods $l \neq k$, and if all consumers' budget constraints are binding, then the market for good $k$ also clears.
\end{lemma}
\begin{pf}
    \textbf{Proof:} Adding up the consumers' budget constraints, we obtain
    \begin{align*}
        \sum_{i=1}^{I} p \cdot x_i = \sum_{i=1}^{I} p \cdot \omega_i + \sum_{i=1}^{I} \sum_{j=1}^{J} \theta_{ij}(p \cdot y_j)
    \end{align*}
    Rearranging the terms, we get
    \begin{align*}
        \sum_{l \neq k} p_l \left(\sum_{i=1}^{I} x_{li} - \omega_l - \sum_{j=1}^{J} y_{lj} \right) = - p_k \left(\sum_{i=1}^{I} x_{ki} - \omega_k - \sum_{j=1}^{J} y_{kj} \right)
    \end{align*}
    By clearing markets of goods $l \neq k$, the left hand side eqauls to zero and so does the right hand side ($p_k > 0$). Thus, the market clearing condition holds for the market of good $k$. 

    \hfill $\square$ 
\end{pf}

\subsection*{10.C Partial equilibrium competitive analysis}

Marshallian partial equilibrium analysis envisions the market for one good that consititutes a small fraction of the whole economy. The small size of the market implies two things:
\begin{itemize}
    \item When the expenditure on the good is a small portion of a consumer's total expenditure, only a small fraction of any additional dollar of wealth will be spent on it. That is, we expect the wealth effect to be small.
    \item The small size of the market of interest leads the price of other goods to be approximately unaffected by changes in this market. Therefore, we are jsutified in treating the expenditure on the other goods as a single composite commodity, called the \textit{numeraire}.
\end{itemize}

With the two implications, we proceed to study a simple two-good quasilinear model, where there are two goods: $l$ and the numeraire. We denote consumer $i$'s consumption of good $l$ and the numeraire as $x_i$ and $m_i$, respectively. For each consumer $i$, he/she has a quasilinear utility function as below\footnote{For convenience, we assume the consumption set is $\mathbb{R} \times \mathbb{R}_+$ (i.e. the consumption of numeraire can be negative.) Additionally, we assume that $\phi_i(\cdot)$ is bounded above and twice differentiable, with $\phi'_i(x_i) > 0$ and $\phi''_i(x_i) < 0$ for all $x_i \geq 0$. We normalize $\phi_i(0) = 0$.}:
\begin{align*}
    u_i(m_i,x_i) = m_i + \phi_i(x_i)
\end{align*}

In the later discussion, we normalize the price of the numeraire to be 1 and denote the price of good $l$ by $p$. Firms are able to produce $l$ from the numeraire $m$. Formally, the production set of firm $j$ is\footnote{We assume the cost function $c_j(\cdot)$ is twice differentiable with $c'_j(q_j) > 0$ and $c''_j(q_j) \geq 0$ for all $q_j \geq 0$. We can think of $c_j(q_j)$ as arising from some multiple-input cost function $c_j(\bar{w}, q_j)$, given the fixed vector of factor prices $\bar{w}$.}:
\begin{align*}
    Y_j = \{(-z_j, q_j): q_j \geq 0,\ z_j \geq c_j(q_j) \},\ z_j:\ \text{the input of good $m$}
\end{align*}

For simplicity, we assume that there is no initial endowment of good $l$ so that all consumption of $l$ must be produced by the firms. We denote $i$'s initial endowment of $m$ by $\omega_{mi} > 0$ and let $\omega_m = \sum_i \omega_{mi}$.\\

Applying def. 10.B.3 and lemma 10.B.1, we conclude that the the allocation $(x^*_1,...x^*_I, q^*_1,q^*_J)$ and price $p^*$ constitute a competitive equilibrium if and only if 
\begin{align*}
    p^* \leq c'_j(q^*_j),\ &\text{with equality if}\ q^*_j > 0 \quad \forall\ j\\
    \phi'(x^*_i) \leq p^*,\ &\text{with equality if}\ x^*_ji > 0 \quad \forall\ i\\
    &\sum_{i=1}^{I} x^*_i = \sum_{j=1}^{J}q^*_j
\end{align*}

The three neccesary and sufficient conditions imply that the equilibrium allocation and price are independent of the distribution of endowments and the ownership shares. (This is the result of the quasi-linear preference!) In fact, the quasi-linear assumption gives rise to an unlimited unit-for-unit transfer of utility across all consumers throught transfers of the numeraire.

\subsubsection*{Comparative statics}

\subsection*{10.D The fundamental welfare theorems in a partial equilibrium context}

Following the quasi-linear property, we know that the set of utilities that can be attained by appropriately distributing the numeraire is:
\begin{align*}
    \{(u_1,...u_I): \sum_{i=1}^{I}u_i \leq \sum_{i=1}^{I}\phi_i(\bar{x}_i) + \omega_m - \sum_{j=1}^{J}c_j(\bar{q}_j) \}
\end{align*}

We may obtain the \textit{the optimal consumption and production levels for good $l$} by solving 
\begin{align*}
    \max_{(x_1,...,x_I), (y_1,...y_J)\geq 0}\ &\sum_{i=1}^{I} \phi_i(x_i) - \sum_{j=1}^{J} c_j(q_j) + \omega_m\\
    &\text{s.t.}\ \sum_{i=1}^{I} x_i - \sum_{j=1}^{J} = 0
\end{align*}

Applying the Lagrange (the multiplier is denoted by $\mu$), we find the following conditions:
\begin{align*}
    \mu \leq c'_j(q^*_j),\ &\text{with equality if}\ q^*_j > 0 \quad \forall\ j\\
    \phi'(x^*_i) \leq \mu,\ &\text{with equality if}\ x^*_ji > 0 \quad \forall\ i\\
    &\sum_{i=1}^{I} x^*_i = \sum_{j=1}^{J}q^*_j,
\end{align*}
which is exactly def 10.B.3 with $p^*$ replaced by the Lagrange multiplier $\mu$\footnote{This implies that the competitive price is exactly the shadow price on the resource constraint.}. And this leads to \textit{the first fundamental theorem of welfare economics}.

\begin{proposition}{10.D.1}
    (\textit{The first fundamental theorem of welfare economics}) If the price $p^*$ and the allocation $(x^*_1,...x^*_I,q^*_1,...,q^*_J)$ constitute a competitive equilibrium, then the allocation is Pareto optimal.
\end{proposition}

We can also develop a converse of the first fundamental theorem of welfare economics, called \textit{the second fundamental theorem of welfare economics}. Note that the equilibrium price, consumption and production levels are unaffected by changes in consumers' wealth level. Hence, a transfer of one unit of the numeraire from $i$ to $i'$ causes their consumption of the numeraire by exactly the same amount of transfer and no other change. Hence, by properly transfering the endownment of the numeraire, we may attain any utility vector in the boundary of the utility possibility set.

\begin{proposition}{10.D.2}
    (\textit{The second fundamental theorem of welfare economics}) For any Pareto optimal levels of utility $(u^*_1,...,u^*_I)$, there are transfers of the numeraire $(T_1,...,T_I)$ satisfying $\sum_{i=1}^{I}T_i = 0$, s.t. a competitive equilibrium reached from the endowment $(\omega_{m1}+T_1,..., \omega_{mI}+T_I)$ yields precisely the utility levels $(u^*_1,...,u^*_I)$.
\end{proposition}


\newpage

\section*{Ch16: Equilibrium and its basic welfare properties}
\hfill

\subsection*{16.B The basic model and definitions}

As in chapter 10, in this chapter, we consider an economy with $I$ consumers, $J$ firms, and $L$ commodities. With our usual notations, such an economy can be charaterized by: 
\begin{align*}
    (\{(X_i, \succcurlyeq_i)_{i=1}^I\}, \{Y_j\}_{j=1}^J, \bar{\omega}),
\end{align*}
where $X_i, Y_j, \bar{\omega} \in \mathbb{R}^L$ and $\succcurlyeq_i$ is a rational preference relation on $X_i$.

\begin{notebox}{Recall of Ch15}
    In chapter 15, we are introduced to the useful tool of \textit{edgeworth box} in a \textit{pure exchange economy} (i.e. there is no production involved). Here we quickly review some definitions.\\
    
    Note that in general equilibrium context, the wealth level of a consumer is not exogenously given but rather determined by the prices and his/her initial endowment. Hence, given the endowment $\omega_i$, consumer $i$'s \textit{budget set} is a function of prices:
    \begin{align*}
        B_i(p) = \{x_i \in \mathbb{R}_+^2: p \cdot x_i \leq p \cdot \omega_i \}
    \end{align*}     
\end{notebox}

\begin{definition}{15.B.1}
     A \textit{Walrasian equilibrium} for an Edgeworth box economy is a price vector $p^* \in \mathbb{R}^L$ and an allocation $(x^*_1, x^*_2)$ s.t. for each consumer $i=1,2$, 
    \begin{align*}
        x^*_i \succcurlyeq_i x'_i\ \text{for all}\ x'_i \in B_i(p^*)
    \end{align*}
\end{definition}

Note that consumers' demand is homogeneous of degree zero in prices $p = (p_1, p_2)$. As a result, only the relative price is determined in an equilibrium.

\begin{definition}{15.B.2}
    An allocation $x$ in the Edgeworth box economy is \textit{Pareto optimal} if there is no other allocation $x'$ with $x'_i \succcurlyeq_i x_i$ for $i =1,2$ and $x'_i \succ_i x_i$ for some $i$.
\end{definition}

The two definitions above can be generalized to the economy in this chapter with production. 
\begin{definition}{16.B.1}
    An allocation $(x,y)$ is a specification of consumption vectors $x_i$ and production vectors $y_j$. The allocation $(x,y)$ is \textit{feasible} if
    \begin{align*}
        \sum_{i=1}^{I} x_i = \bar{\omega} + \sum_{j=1}^{J} y_j
    \end{align*}
We denote the set of feasible allocations by $A$.
\end{definition}
\begin{definition}{16.B.2}
    An allocation $(x,y)$ is \textit{Pareto optimal} if there is no other allocation $(x',y') \in A$ 
    that \textit{Pareto dominates} it. That is, if there is no feasible allocation $(x',y')$ s.t. $x'_i \succcurlyeq_i x_i$ for all $i$ and $x'_i \succ_i x_i$ for some $i$.
\end{definition}

\subsubsection*{Private ownership economy}

In an private ownership economy, all goods are traded in the market with publicly known prices that consumers and firms take as unaffected by their own actions. Each consumer also has a claim $\theta_i$ of the firms' profits. Hence, we characterize such an economy by:
\begin{align*}
    (\{(X_i, \succcurlyeq_i)\}, \{Y_j\}, \{(\omega_i, \theta_{i1}, ... \theta_{iJ}) \})
\end{align*}

\begin{definition}{16.B.3}
    Given a private ownership economy, an allocation $(x^*,y^*)$ and a price vector $p^*$ is a \textit{Walrasian equilibrium} if:
    \begin{itemize}
        \item [(i)] For every firm $j$, $y^*_j$ maximizes profits in $Y_j$. That is,
        \begin{align*}
            p^* \cdot y_j \leq p^* \cdot y^*_j\ \text{for all}\ y_j \in Y_j
        \end{align*}
        \item[(ii)] For every consumer $i$, $x^*_i$ is maximal for $\succcurlyeq_i$ in the budget set
        \begin{align*}
            \{x_i \in X_i: p \cdot x_i \leq p \cdot \omega_i + \sum_j \theta_{ij}p \cdot y^*_i \}
        \end{align*}
        \item[(iii)] Market clearing: $\sum_i x^*_i = \bar{\omega} + \sum_j y^*_j$
    \end{itemize}
\end{definition}

\subsubsection*{Price equilibrium with transfers}

\begin{definition}{16.B.4}
    Given an economy, an allocation $(x^*,y^*)$ and a price vector $p = (p_1,...p_L)$ form a \textit{price equilibrium with transfers} if there exists an assignment of wealth $(w_1,...w_I)$ with $\sum_i w_i = p \cdot \bar{\omega} + \sum_j p \cdot y^*_j$ s.t.:
    \begin{itemize}
        \item[(i)] For every firm $j$, $y^*_j$ maximizes profits in $Y_j$. That is,
        \begin{align*}
            p^* \cdot y_j \leq p^* \cdot y^*_j\ \text{for all}\ y_j \in Y_j
        \end{align*}
        \item[(ii)] For every consumer $i$, $x^*_i$ is maximal for $\succcurlyeq_i$ in the budget set
        \begin{align*}
            \{x_i \in X_i: p \cdot x_i \leq w_i \}
        \end{align*}
        \item[(iii)] Market clearing: $\sum_i x^*_i = \bar{\omega} + \sum_j y^*_j$  
    \end{itemize}
\end{definition}

\noindent \textbf{Remark}: Note that the notion of price equilibrium with transfers requires only that there be some wealth distribution s.t. $(x^*,y^*)$ and price vector $p$ form an equilibrium. As a result, the Walrasian equilibrium is a special case of price equilibrium with transfers.


\subsection*{16.C The first fundamental theorem of welfare economics}

\begin{definition}{16.C.1}
    The preference relation $\succcurlyeq_i$ on $X_i$ is \textit{locally nonsatiated} if for every $x_i \in X_i$ and every $\epsilon > 0$, there is a $x'_i \in X_i$ s.t. $\|x'_i - x_i\| < \epsilon$ and $x'_i \succ_i x_i$.
\end{definition}

\begin{exercise}{16.C.1}
    Show that if a consumption set $X_i \subseteq \mathbb{R^L}$ is non-empty, closed, and bounded, and the preference relation $\succcurlyeq_i$ on $X_i$ is continuous, then $\succcurlyeq_i$ can not be locally nonsatiated.
\end{exercise}
\begin{pf}
    \textbf{Proof}: Since $X_i \subseteq \mathbb{R}^L$ is non-empty, closed, and bounded, it is compact.\\
    
    As $\succcurlyeq_i$ is continuous, it admits a continuous utility representation $u_i(\cdot)$ on $X_i$. Hence, we know there exists a maximal element $x_i \in X_i$. Hence, there exists no $x'_i \in X_i$ s.t. $x'_i \succ x_i$. That is, $\succcurlyeq_i$ is not locally nonsatiated.

    \hfill $\square$
\end{pf}

A direct result of the exercise is that if $\succcurlyeq_i$ is continuous and locally nonsatiated, then any closed $X_i$ must be unbounded.

\begin{proposition}{16.C.1}
    (\textit{The first fundamental theorem of welfare economics}) If preferences are locally nonsatiated and if $(x^*,y^*,p)$ is a price equilibrium with transfers, then the allocation $(x^*, y^*)$ is Pareto optimal. In particular, any Walrasian equilibrium allocation is Pareto optimal.
\end{proposition}
\begin{pf}
    \textbf{Proof}: Let $(x^*,y^*)$ be a price equilibrium with transfers and the associated wealth levels are $w_1,...w_I$.\\
    
    Suppose by contradiction, there exists a feasible allocation $(x,y)$ that Pareto dominates $(x^*,y^*)$. We know that
    \begin{align*}
        x_i \succcurlyeq_i x^*_i\ \text{for all}\ i,\ \text{and}\ x_i \succ_i x^*_i\ \text{for some}\ i
    \end{align*}
    By def 16.B.4, we know that if $x_i \succ_i x^*_i$, then $p \cdot x_i > w_i$. That is, $x_i$ must be unaffordable for consumer $i$.\\

    Now to apply this further, we claim the following (exercise 16.C.2):
    \begin{claim}{}
        Suppose $\succcurlyeq_i$ is a locally nonsatiated preference relation and $x^*_i$ is maximal in $\{x_i \in X_i: p \cdot x_i \leq w_i\}$. If $x_i \succcurlyeq_i x^*_i$ then $p \cdot x_i \geq w_i$.
    \end{claim}
    \begin{pf}
        \textbf{Proof}: Suppose by contradiction that $x_i \succcurlyeq_i x^*_i$ and $p \cdot x_i < w_i$.\\

        As $\{x_i \in X_i: p \cdot x_i < w_i \}$ is an open set, we can find $\epsilon > 0 $ s.t. 
        \begin{align*}
            B(x_i, \epsilon) \subseteq  \{x_i \in X_i: p \cdot x_i < w_i \}
        \end{align*}

        Fix such $\epsilon > 0$. Local nonsatiation implies there exists $x'_i \in B(x_i, \epsilon)$ s.t. $x'_i \succ x_i$. By transitivity ($\succcurlyeq_i$ is rational), we know $x'_i \succ x^*_i$ and $p \cdot x'_i < w_i$. Hence, we conclude that $x^*_i$ is not maximal in the feasible consumption set, which is a contradiction. We conclude that $x_i \succcurlyeq_i x^*_i$ implies $p \cdot x_i \geq w_i$.

        \hfill $\square$
    \end{pf}
    This claim simply states that if a bundle $x_i$ is at least as good as the optimal bundle $x^*_i$, then it is at best just affordable.\\

    Consider an allocation $(x,y)$ that Pareto dominates $(x^*,y^*)$ (i.e. $x_i \succcurlyeq_i x^*_i$ for all $i$ and $x_i \succ_i x^*_i$ for some $i$). We must have $p \cdot x_i \geq w_i$ for all $i$ and $p \cdot x_i > w_i$ for some $i$.
\end{pf}   
\begin{pf}{}
    Hence, we have 
    \begin{align*}
        \sum_i p \cdot x_i > \sum_i w_i = p \cdot \bar{\omega} + \sum_j p \cdot y^*_j
    \end{align*}
    Note that by def 16.B.4, $y^*_j$ maximizes profits for all $j$. Thus, we obtain
    \begin{align*}
        \sum_i p \cdot x_i > p \cdot \bar{\omega} + \sum_j p \cdot y^*_j \geq p \cdot \bar{\omega} + \sum_j p \cdot y_j
    \end{align*}
    This shows that $(x,y)$ is not feasible and as a result, $(x^*,y^*)$ is Pareto optimal.

    \hfill $\square$
\end{pf}

\subsection*{16.D The second fundamental theorem of welfaere economics}









\end{document}
